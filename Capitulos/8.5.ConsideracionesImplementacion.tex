\section{Consideraciones sobre la implementación}
En este capitulo se describirán tanto las causas que han llevado a adoptar la previa especificación del diseño del sistema como las razones detrás de ellas. Con el objetivo de ayudar a entender el porqué de las decisiones tomadas se hablará sobre la privacidad, seguridad, la combinación de los modelos y las limitaciones técnicas presentes en este proyecto.
\subsection{Privacidad}
Con el objetivo de no vulnerar la privacidad de los datos, la agregación de los modelos será realiza por consenso sobre usuarios sintéticos. De esta forma, se evitan problemas derivados del robo de datos y las consecuencias tanto legales como personales de este.
\\ \\
Partiendo del principio de que si no circula ningún dato de ningún usuario por la red no se corren tantos riesgos se plantea este sistema de agregación como una alternativa viable a los sistemas tradicionales de ensamblado.

\subsection{Seguridad}
Este sistema está expuesto a los ataques de seguridad derivados de la implementación del Federated Learning. Sin embargo, debido a la naturaleza del algoritmo de consenso, algunos ataques podrían ser minimizados o incluso erradicados.
\\ \\
Un ejemplo de ello es el ataque por envenenamiento. Existen dos tipos de envenenamiento, por datos y por modelo, pero ambos consisten en que un participante de la red, con el objetivo de dañar tanto al sistema como a los participantes, envía modelos de inteligencia artificial (o los datos si es por envenenamiento de datos) debidamente manipulados. En los sistemas de agregación de modelos de inteligencia artificial por ensamblado este tipo de ataque suele ser bastante dañino, ya que es capaz de distorsionar por completo los resultados del modelo final con lo enviado por el atacante.
\\ \\
Sin embargo, debido a la condición del sistema de agregación por consenso, este tipo de ataques no son tan efectivos, ya que el modelo con datos envenenados podría ser ignorado si el método por el que se realiza el consenso es la moda. Con la moda se elegiría el ranking en función de los valores más frecuentes, es decir, el modelo del atacante sería ignorado por presentar predicciones que no se ajustan a ninguno de los demás participantes.
\\ \\
Cabe destacar que con el objetivo de detectar ataques y encontrar posibles modelos maliciosos se podrían descartar del proceso todos aquellos modelos que presenten unas predicciones totalmente diferentes. Se podría analizar la varianza de las predicciones de cada modelo para descartar del proceso a presuntos modelos envenenados.

\subsection{Combinación de modelos}
Debido a que el principal objetivo de este proyecto es el desarrollo de un sistema de recomendación con aprendizaje federado con la privacidad como patron de diseño, la combinación de los diferentes modelos de los participantes ha sido un factor muy importante a tener en cuenta. En el Machine Learning tradicional esta agregación de modelos se producía mediante el ensamblado de modelos. Sin embargo, este proceso no es compatible con este proyecto puesto que para poder ensamblar los modelos habría que compartir información de los participantes de la red.
\\ \\
Para evitar compartir ningún tipo de información que comprometiese la seguridad y privacidad de los participantes de la red de aprendizaje federado pero conseguir que los participantes aprendiesen entre ellos, se optó por compartir los modelos de inteligencia artificial exclusivamente.

\subsection{Limitaciones técnicas}
Debido a que los dispositivos que iban a formar la red iban a ser 4 Rasberry Pis modelo 3b+, el procesado de los datos era un proceso de especial importancio por la \textit{escasa} capacidad de computo de estos dispositivos. Estos dispositivos, como se puede ver en el capítulo de especificación del diseño, son dispositivos que tienen cuatro núcleos y 1.4GHz de frecuencia, lo que comparado con un ordenador de sobremesa actual supone una gran diferencia de potencia de cálculo.
\\ \\
Además, esa no es la unica diferencia que tienen respecto a los ordenadores de escritorio, las Raspberry Pis funcionan sobre arquitectura ARMv8 y no sobre las tan usadas x64 y x86 . Esto implica un gran problema a la hora de instalar los paquetes necesarios para que el proyecto sea ejecutable en estos dispositivos, ya que la mayoría de los paquetes necesarios están pensados para las arquitecturas más comunes (x64 y x86). Sin embargo, se han podido encontrar las versiones funcionales de todos los paquetes para estos dispositivos.
\\ \\
El dispositivo que iba a agregar los datos, la Nvidia Jetson Nano, ha sido más sencilla de configurar puesto que esta funciona sobre arquitectura AArch64, extensión de 64bits de la arquitectura ARM.
