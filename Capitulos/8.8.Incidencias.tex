\section{Incidencias ocurridas}
Durante el desarrollo del proyecto ocurrieron varios problemas con los dispositivos implicados que se han decidido mencionar. 
\\ \\
Al principio del desarrollo del proyecto se optó por la utilización de otro modelo de Raspberries, menos potentes y más antiguas, que eran incompatibles con las versiones de los paquetes que había que utilizar. Tras muchas pruebas y después de intentar compilar uno a uno cada paquete para los dispositivos se decidió desistir y cambiarlos por las actuales Raspberry Pi 3B +, mucho más recientes.
\\ \\
Al realizar el cambio de los cuatro dispositivos y tener que reconfigurar todas las tarjetas SD se perdió gran parte del tiempo, puesto que las primeras fases de la configuración han de realizarse con la Raspberry Pi conectada a una pantalla con teclado y ratón para habilitar los diferentes protocolos de comunicación. Una vez con esta configuración, se ejecutaron los \textit{script}s de aprovisionamiento preparados para reinstalar todos los paquetes, lo que permitió que el proceso no se alargara mucho más. 
\\ \\
Cuando parecía que todo estaba solucionado, una Raspberry Pi falló y se rompió, lo que provocó que hubiera que desmontar todo el rack \ref{fig:RackRaspberry} y poner una Raspberry Pi nueva. Este proceso al tratarse únicamente de hardware y al intercambiarse por el mismo modelo no trajo mayores problemas puesto que se podía utilizar la misma tarjeta SD con la misma configuración.
