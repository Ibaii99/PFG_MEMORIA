\chapter*{Resumen}

\thispagestyle{fancy}

Este proyecto final de grado presenta un estudio sobre la privacidad en el aprendizaje federado mediante el desarrollo de un sistema de recomendación. El sistema de recomendación de estrategias de persuasión parte del proyecto de R. Sánchez y col.\autocite{sanchez-corcueraPersuasionbasedRecommenderSystem2020} y será modificado para operar de forma descentralizada en varios dispositivos. Cada dispositivo elaborará su propio modelo de inteligencia artificial con sus datos. Mediante el protocolo de Federated Learning los modelos de inteligencia artificial de los distintos dispositivos serán combinados para mejorar su precisión en las recomendaciones. El enfoque de agregación de los modelos de inteligencia artificial será interpretado de forma diferente al canónico Federated Learning, en este proyecto se abordará el tema de la agregación desde el reentrenado de los modelos con información sintética, evitando así, compartir información sensible de los dispositivos.
\\\\
El estudio analizará el rendimiento del sistema de recomendación funcionando sobre aprendizaje automático de forma centralizada, es decir con toda la información disponible, así como el rendimiento del mismo sistema con información distribuida, tanto antes como después de ser agregada en el servidor de Federated Learning. De esta forma, se podrán observar las ventajas de la segregación de información y el impacto del Federated Learning en los modelos de los participantes, tanto a nivel de rendimiento como de mejora de la privacidad.
\\\\
El estudio y los experimentos se realizarán sobre una red real, creada sobre un entorno de desarrollo basado en varias Raspberries Pi (para representar a los participantes de la red de Federated Learning) y una NVIDIA Jetson Nano (como servidor central para agregar los modelos).
\\\\
\textbf{Descriptores}\\
Inteligencia Artificial, Machine Learning, Federated Learning, Sistema de Recomendación.