\chapter{Presupuesto}
\thispagestyle{fancy}
\fancyhead[LE]{\thechapter.Presupuesto} 
El presupuesto de este proyecto cuenta con dos apartados, el presupuesto de los materiales y el de los recursos humanos. Adicionalmente se ha añadido un apartado para explicar los requisitos de la plataforma de desarrollo, que al no ser adquirida para este proyecto no entra dentro de los costes, pero es un elemento fundamental para el desarrollo de software en este proyecto. 

\section{Plataforma de desarrollo}
Aunque el proyecto conste únicamente del gasto en materiales y recursos humanos se ha utilizado un ordenador de escritorio para su desarrollo. Aunque el presupuesto de este ordenador no sea exclusivo para el proyecto ha sido necesario para su desarrollo y para agilizar los despliegues y las pruebas. Además, este ordenador ha sido utilizado para desarrollar el código y probarlo, actividad que también ha agilizado mucho el desarrollo del proyecto.
\\ \\
Es por esto que, aunque no entre en el presupuesto, merecía ser mencionado. El ordenador cuenta con los diferentes componentes, tabla \ref{tab:PlataformaDesarrollo}.

\begin{table}[H]
    \begin{center}
    %Se centra la tabla.
        \begin{tabular}{|c|}
            % -------------
            \hline
            \rowcolor{Cyan} 
            \textbf{Plataforma de desarrollo} \\ 
            % -------------
            \hline
            \rowcolor{GrisTabla}
            \textbf{Ordenador de torre} \\
            \hline
            GPU: Nvidia GTX 970 \\
            \hline
            CPU: Ryzen 5 3600 \\
            \hline
            RAM: 16 gb \\
            % -------------
            \hline
        \end{tabular}
        \caption{\centering Presupuesto de los materiales utilizados.}
        \label{tab:PlataformaDesarrollo}
    \end{center}    
\end{table}


\newpage
\section{Materiales}
En cuanto al presupuesto de los materiales que se han utilizado en el proyecto, se ha elaborado la tabla \ref{tab:PresupuestoMateriales} donde se detalla tanto la lista de materiales, como su precio.
\\ \\
Este presupuesto está agrupado en 5 categorías, las Raspberries Pi, las tarjetas SD, la Jetson Nano, los cables y los elementos de telecomunicación. Esta clasificación permite observar en qué grupos se ha gastado más. 

\begin{table}[H]
    \begin{center}
    %Se centra la tabla.
        \begin{tabular}{|r|c|r|r|}
            % -------------
            \hline
            \rowcolor{Cyan} 
            % -------------
            \multicolumn{4}{|c|}{\textbf{Materiales}}\\
            \hline
            \textbf{Hardware} & \textbf{Cantidad} & \textbf{Precio unidad}& \textbf{Precio total}\\ 
            % -------------
            \hline
            \rowcolor{GrisTabla}
            \textbf{Raspberries} &  \textbf{5} & & \textbf{185,40 \euro}\\
            Pi 3 B & 1 & 37,44 \euro & 37,44 \euro\\
            Pi 3 B+ & 4 & 36,99 \euro & 147,96 \euro\\
            % -------------
            \hline
            \rowcolor{GrisTabla}
            \textbf{SD-s} &  \textbf{8} & & \textbf{50 \euro}\\
            16 GB & 1 & 6 \euro & 42 \euro\\
            32 GB & 4 & 8 \euro & 8 \euro\\
            % -------------
            \hline
            \rowcolor{GrisTabla}
            \textbf{Jetson Nano} & \textbf{1} & & \textbf{140 \euro}\\
            NVIDIA Jetson Nano & 1 & 140 \euro & 140 \euro\\
            % -------------
            \hline
            \rowcolor{GrisTabla}
            \textbf{Cables} &  \textbf{5} & & \textbf{137 \euro}\\
            Alimentación Raspberry & 5 & 10 \euro & 50\euro\\
            Alimentación Jetson Nano & 1 & 15 \euro & 15 \euro\\
            Regleta enchufes & 2 & 10 \euro & 10 \euro\\
            Rack Raspberries & 1 & 20 \euro & 20 \euro\\
            RJ45 & 8 & 4 \euro & 32 \euro\\
            % -------------
            \hline
            \rowcolor{GrisTabla}
            \textbf{Telecomunicaciones} &  \textbf{2} & & \textbf{29,41} \euro\\
            Switch & 1 & 11 \euro & 11 \euro\\
            Router & 1 & 18,41 \euro & 18,41 \euro\\
            % -------------
            \hline
            \rowcolor{Naranja} 
            % -------------
            \multicolumn{3}{|r}{\textbf{Total}} & \textbf{541,81 \euro}\\
            \hline
            % \rowcolor{Naranja} 
            % \textbf{V} & \textbf{elit} \\ \hline
        \end{tabular}
        \caption{\centering Presupuesto de los materiales utilizados.}
        \label{tab:PresupuestoMateriales}
    \end{center}    
\end{table}


\newpage
\section{Recursos Humanos}
En cuanto a los recursos humanos que intervienen en el proyecto se ha decidido contar con los siguientes tipos de especialistas que, desempeñando estos roles, desarrollarán el proyecto.

\begin{table}[H]
    \begin{center}
    %Se centra la tabla.
        \begin{tabular}{|l|c|p{0.4\linewidth}|}
            % -------------
            \hline
            \rowcolor{Cyan} 
            \textbf{Perfiles} & \textbf{Precio hora} & \textbf{Responsabilidades} \\ 
            % -------------
            \hline
            \rowcolor{GrisTabla}
            & & $\bullet$ Organización\\
            \rowcolor{GrisTabla}
            & & $\bullet$ Gestión\\
            \rowcolor{GrisTabla}
            & & $\bullet$ Delimitar alcance y objetivos\\
            \rowcolor{GrisTabla}
            \multirow{-4}{*}{\textbf{Jefe de proyecto}} & \multirow{-4}{*}{\textbf{45 \euro}} & $\bullet$ Búsqueda de tecnologías\\
            \hline

            & & $\bullet$ Realización de la memoria técnica\\
            & & $\bullet$ Preparación de la defensa\\
            & & $\bullet$ Entre e impresión de la memoria y defensa\\
            \multirow{-4}{*}{\textbf{Analista de software interno}} & \multirow{-4}{*}{\textbf{30 \euro}} & $\bullet$ Búsqueda de nuevas tecnologías\\
            \hline

            \rowcolor{GrisTabla}
            & & $\bullet$ Ideas sobre cómo afrontar problemas\\
            \rowcolor{GrisTabla}
            & & $\bullet$ Resolver dudas\\
            \rowcolor{GrisTabla}
            & & $\bullet$ Presentar sistema de recomendación anterior\\
            \rowcolor{GrisTabla}
            \multirow{-4}{*}{\textbf{Analista de software externo}} & \multirow{-4}{*}{\textbf{30 \euro}} & $\bullet$ Consejos\\
            \hline

            & & $\bullet$ Idear la estructura del modelo de consenso\\
            & & $\bullet$ Desarrollar el protocolo de federated learning\\
            \multirow{-3}{*}{\textbf{Arquitecto de software}} & \multirow{-3}{*}{\textbf{45 \euro}} & $\bullet$ Planificar conexiones entre dispositivos y requisito\\
            \hline

                
            % -------------
            \rowcolor{GrisTabla}
            \textbf{Desarrollador de software} & \textbf{35 \euro} & $\bullet$  Programación del proyecto\\
            \hline

            & & $\bullet$ Revisar gramática de la memoria\\
            & & $\bullet$ Revisar el software desarrollado\\
            \multirow{-3}{*}{\textbf{Gestor de calidad}} & \multirow{-3}{*}{\textbf{30 \euro}} & $\bullet$ Revisión de bugs y errores\\
            \hline

        \end{tabular}
        \caption{\centering Precio hora de los distintos perfiles que intervienen en el proyecto.}
        \label{tab:PrecioPerfil}
    \end{center}    
\end{table}
\pagebreak
Estos roles de la tabla \ref{tab:PrecioPerfil} serán interpretados por los siguientes colaboradores del proyecto, presentes en la tabla \ref{tab:PersonaPerfil}. Cabe destacar que el rol de analista interno solo será llevado a cabo por Ibai Guillén, el resto de participantes, en caso de desempeñarlo, ejercerán de analistas externos.
\begin{table}[H]
    \begin{center}
        %Se centra la tabla.
        \begin{adjustbox}{angle=90}  
                \begin{tabular}{|p{0.22\linewidth}|p{0.12\linewidth}|p{0.12\linewidth}|p{0.15\linewidth}|p{0.19\linewidth}|p{0.135\linewidth}|}
                    % -------------
                    \hline
                    \rowcolor{Cyan} 
                    \backslashbox{Personas}{Perfiles} & \textbf{Jefe de proyecto} & \textbf{Analista} & \textbf{Arquitecto} & \textbf{Desarrollador} & \textbf{Gestor de calidad}\\ 
                    % -------------
                    \hline
                    \rowcolor{GrisTabla}
                    \textbf{Oihane Gómez} &  & 2 horas &  &  & \\
                    % -------------
                    \hline
                    \textbf{Rubén Sánchez} &  & 3 horas &  &  & \\
                    % -------------
                    \hline
                    \rowcolor{GrisTabla}
                    \textbf{Diego Casado} & 6 horas & 8 horas & 6 horas &  & 4 horas\\
                    % -------------
                    \hline
                    \textbf{Ibai Guillén} & 40 horas & 100 horas & 50 horas & 110 horas & 32 horas\\
                    \hline
                \end{tabular}
        \end{adjustbox}  
        \caption{\centering Horas de cada persona en cada labor} \label{tab:PersonaPerfil}
  
    \end{center}
\end{table}%

El presupuesto para esta plantilla sería el producto de las horas que han dedicado las personas de la tabla \ref{tab:PersonaPerfil} con el precio hora de la tabla \ref{tab:PrecioPerfil}. Esto queda reflejado en la siguiente tabla \ref{tab:PresupuestoRRHH}.

\begin{table}[H]
    \begin{center}
    %Se centra la tabla.
        \begin{tabular}{|c|c|}
            % -------------
            \hline
            \rowcolor{Cyan} 
            \textbf{Persona} & \textbf{Coste total} \\ 
            % -------------
            \hline
            \rowcolor{GrisTabla}
            Oihane Gómez & 60 \euro\\
            \hline
            Rubén Sánchez & 90 \euro\\
            \hline
            \rowcolor{GrisTabla}
            Diego Casado & 900 \euro\\
            \hline
            Ibai Guillén & 15.010 \euro\\
            \hline
            \rowcolor{Naranja}
            \textbf{Total} & \textbf{16.060 \euro}\\
            \hline
            % -------------
        \end{tabular}
        \caption{\centering Presupuesto total de RRHH.}
        \label{tab:PresupuestoRRHH}
    \end{center}    
\end{table}
