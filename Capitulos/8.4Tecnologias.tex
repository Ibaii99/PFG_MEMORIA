\section{Tecnologías utilizadas}

\subsection{Sistema de recomendación}
El sistema de recomendación de R.Sánchez utiliza LightFM\footnote{LightFM es una implementación en python de varios algoritmos populares de recomendación.} para crear los modelos de inteligencia artificial. Para implementar el aprendizaje federado se ha tenido que modificar gran parte del sistema de recomendación previo y se ha tenido que hacer uso de las utilidades que proporciona LightFM que se van a explicar en los siguientes apartados. 
\subsubsection{Creación del modelo}
El paso de creación del modelo es un paso complejo puesto que conlleva la correcta gestión de muchos datos e información. Asimismo, estos datos han de estar correctamente estructurados y construidos, puesto que LightFM requiere que la información este convertida a matrices dispersas y no tolera elementos vacíos en ellas. Para ello LightFM proveé de herramientas de creación de datasets que facilitan la creación de las matrices, de forma que sea más fácil incluirlas en el modelo. 
\\ \\
Entre estas herramientas se encuentran:

\begin{itemize}
    \item \textit{build\_user\_features($\ldots$)} \quad Permite crear la matriz CSR de usuarios y atributos de usuarios.
    \item \textit{build\_item\_features($\ldots$)} \quad Permite crear la matriz CSR  de items y atributos de items.
    \item \textit{build\_interactions($\ldots$)} \quad Permite crear las matriz COO de interacciones y las matriz COO de sus correspondientes pesos.
\end{itemize}

Hay que tener en cuenta que las matrices CSR (Compressed Sparse Row) hacen referencia a las matrices que admiten operaciones matriciales y acceso eficientemente; y que las matrices COO (Coordinate list) hacen referencia a las matrices que soportan modificaciones eficientemente, son generalmente utilizadas para construir matrices.


\paragraph{Atributos de los usuario}
Para crear el modelo es necesario, entre otras cosas, disponer de las características de los usuarios. En este caso contamos con multitud de atributos sobre cada usuario: edad, género, nivel educativo, país, cultura de trabajo, perfil PST (Pinball, Shortcut, Thought-ful), barreras, intenciones y confianza.
\\ \\
Para convertir toda esa información del usuario a la matriz CSR que exige LightFM habrá que llamar a \textit{build\_user\_features($\ldots$)} con los IDs de usuario y y sus atributos, formando una lista que tenga listas de los IDs de los usuarios con sus listas de atributos, es decir:
\begin{align*}
    \begin{bmatrix}
        \begin{bmatrix} 
            userId$$_{1}$$ & \begin{bmatrix} feature$$_{11}$$ &\cdots & feature$$_{1w}$$ \end{bmatrix}$$_1$$
        \end{bmatrix}
        &
        \ldots
        &
        \begin{bmatrix} 
            userId$$_{v}$$ & \begin{bmatrix} feature$$_{v1}$$ &\cdots & feature$$_{vw}$$ \end{bmatrix}$$_v$$
        \end{bmatrix}
    \end{bmatrix}
\end{align*}
Teniendo en cuenta que:
\begin{align*}
    \textit{v}\gets & \textit{Cantidad de IDs de usuario }
    &&\\
    \textit{w} \gets & \textit{Contidad de atributos por usuario}
    &&\\
    \textit{userId$_{v}$} \gets & \textit{ID del usuario v}
    &&\\
    \textit{feature$_{vw}$} \gets & \textit{Atributo w del usuario v}
\end{align*}
%
Una vez creada la lista y pasado como parámetro al método, obtendremos la matriz CSR de atributos de usuario.

\paragraph{Atributos de los elementos}
Además de los usuarios y de sus atributos, también se ha de disponer de los elementos a ordenar en el ranking, llamados items, y de sus características. En este caso estos items representan las estrategias de persuasión por las que se preguntó a los usuarios en el cuestionario. Sin embargo, al contrario que en el punto anterior, no contamos con tantos atributos sobre estos items, sino que cada estrategia de persuasión cuenta con dos atributos llamados dimensiones. Estos atributos no se encuentran presente en todas las estrategias, lo que da como resultado que haya algunas que tengan dos, una o ninguna dimension.
\\ \\
Para convertir toda esa información de los items a la matriz CSR que exige LightFM habrá que llamar a \textit{build\_item\_features($\ldots$)} con los IDs de las estrategias y y sus atributos, formando una lista que tenga listas de los IDs de las estrategias con sus listas de atributos, es decir:
\begin{align*}
    \begin{bmatrix}
        \begin{bmatrix} 
            itemId$$_{1}$$ & \begin{bmatrix} feature$$_{11}$$ &\cdots & feature$$_{1w}$$ \end{bmatrix}$$_1$$
        \end{bmatrix}
        &
        \ldots
        &
        \begin{bmatrix} 
            itemId$$_{v}$$ & \begin{bmatrix} feature$$_{v1}$$ &\cdots & feature$$_{vw}$$ \end{bmatrix}$$_v$$
        \end{bmatrix}
    \end{bmatrix}
\end{align*}
Teniendo en cuenta que:
\begin{align*}
    \textit{v}\gets & \textit{Cantidad de IDs de estrategias }
    &&\\
    \textit{w} \gets & \textit{Contidad de dimensiones por estrategia}
    &&\\
    \textit{itemId$_{v}$} \gets & \textit{ID de la estrategia v}
    &&\\
    \textit{feature$_{vw}$} \gets & \textit{Atributo w de la estrategia v}
\end{align*}
%
Una vez creada la lista y pasado como parámetro al método, obtendremos la matriz CSR de atributos de usuario.
%
%
\paragraph{Interacciones}


Para crear un modelo de LightFM se necesita

user\_features, item\_features 

model = LightFM() 

% model.fit(train\_interactions,
%         user\_features=user\_features,
%         sample\_weight=train\_weights,
%         epochs=100,
%         num\_threads=4,
%         verbose=False)
% ranks\_test = model.predict\_rank(test\_interactions, 
%         #train\_interactions= train\_interactions, 
%         user\_features=user\_features,
%         check\_intersections=True)                    

\subsubsection{Ajuste}
\subsubsection{Predicción}


\subsubsection{Evaluación}
La evaluación de los resultados obtenidos de las predicciones de los modelos es un apartado fundamental que permite analizar la eficacia de estas predicciones. Además, la utilización de métricas y técnicas permitirá poder comparar los resultados entre sí para poder demostrar si la predicción es correcta, aceptable o errónea.
\paragraph{Métricas}
