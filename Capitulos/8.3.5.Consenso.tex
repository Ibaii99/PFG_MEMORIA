\subsection{Agregación de modelos por consenso}\label{Consenso}
A la hora de agregar los diferentes modelos de los participantes se estudiaron los distintos algoritmos de los sistemas de aprendizaje federado, pero todos tenían algo en común, accedían a los datos del modelo. Aunque estos datos de los modelos sean meras matrices y números, el objetivo de este proyecto era no acceder en ningún momento a la información presente en los modelos de inteligencia artificial de los participantes, por lo que se buscó otra solución.
\\ \\
La solución llegó con la idea de que no hacía falta combinar los distintos modelos para crear uno, sino que más bien, deberían compartir sus formas de operar ante ciertos casos. Esta idea se empezó a explotar mediante la generación de usuarios sintéticos, los cuales cada modelo de cada participante de la red de aprendizaje federado debería predecir su ranking. Cuando cada participante predice el ranking para cada usuario sintético están demostrando cómo actuarían ante ciertas entradas de datos, por lo que si se compartiera este conocimiento los participantes se verían beneficiados de las formas de operar de otros participantes.
\\ \\
El problema es que, desde un punto de vista objetivo, puede que algunas formas de operar no sean del todo precisas, por lo que no se debería compartir los rankings predichos por todos los modelos sin ningún filtrado. Para realizar este filtrado y con el objetivo de llegar a un punto intermedio, donde las predicciones que da cada modelo para cada usuario se combinan y todos los modelos lleguen a un acuerdo para el ranking de un usuario, se creó el modelo de consenso. 
\\ \\
Este modelo de consenso se explicará en los siguientes apartados, donde se explicará cómo se generan los datos sintéticos, como se generan los rankings, las explicaciones matemáticas del consenso y el proceso de reentrenado de los modelos.

\subsubsection{Generar datos personales sintéticos}\label{Consenso:Usuarios_Sinteticos}
Uno de los elementos imprescindibles para poner en práctica el sistema de agregación por consenso es la necesidad de datos de usuarios, que con el objetivo de evitar cualquier violación de privacidad serán creados en el servidor, siendo de ahora en adelante denominados como usuarios sintéticos.
\\ \\
Estos usuarios sintéticos tendrán los mismos atributos que cualquier usuario del sistema, es decir, contarán con todos los atributos presentes en la tabla \ref{tab:AtributosUsuarios}. Como ya se ha dicho antes, el proceso de creación de estos datos es crucial para el correcto funcionamiento del sistema, ya que si los usuarios generados fueran de un perfil concreto el aprendizaje no sería favorable para todos los participantes de la red, puesto que lo más seguro es que únicamente el participante con más usuarios de este perfil sea beneficiado mientras que para los demás participantes supondrá una pérdida de precisión por el ruido introducido en su modelo. Es por eso que se ha tomado la decisión de generarlos aleatoriamente, evitando así estancarse en perfiles concretos y tratando de favorecer a todos los participantes. 
\subsubsection{Generar ranking de los usuarios}
Los datos de los usuarios sintéticos por sí solos no son suficientes ni para entrenar ni para mejorar ningún modelo de inteligencia artificial, hace falta saber el ranking que estos usuarios sintéticos darían a las diferentes estrategias de persuasión, tabla \ref{tab:EstrategiasPersuasion}. 
\\ \\
Esta cuestión no puede ser abordada aleatoriamente como la anterior, ya que cada usuario ordena en base a su opinión y sus preferencias las estrategias de persuasión, por lo cual, de hacerlo aleatoriamente se distorsionaría la realidad y no sería representativo. Además, si este ranking se generase aleatoriamente y se compartiera con los participantes de la red, distorsionaría por completo sus modelos, lo que llevaría a una pérdida inmensa de precisión. 
\\ \\
Es debido a esta problemática que se ha creado el método de consenso. Este método trata de representar la \textit{``opinión''} de cada participante de la red mediante la estimación del ranking para cada usuario sintético, todas estas \textit{``opiniones''} serán puestas en común mediante el algoritmo de consenso, lo que dará como resultado un ranking por usuario sintético que represente a todos los usuarios de la red.
\newpage
\subsubsection{Consenso}
El proceso de consenso comienza con la ya mencionada creación de usuarios sintéticos aleatoriamente, pero la generación del ranking de estos se lleva a cabo de la siguiente manera.

\paragraph{Estructuras básicas} En primer lugar hay que definir la lista de las estrategias \textit{E} a ser ordenadas por los modelos \textit{m} de los participantes, que en este caso serán las mismas que las de la tabla \ref{tab:EstrategiasPersuasion}. Estas estrategias serán representadas como \textit{e$_{j}$}, siendo \textit{j} la cantidad de estrategias de la lista \textit{E} y \textit{e$_{j}$} la estrategia final de esta lista. 
\begin{center}
    \textit{E} = $\begin{bmatrix} \textit{e$_{1}$} & \textit{e$_{2}$} & \textit{e$_{3}$} & \textit{e$_{4}$} & \cdots & \textit{e$_{j}$} \end{bmatrix}$ 
\end{center}
A la hora de crear una predicción de ranking, denotada \textit{r}, para las estrategias \textit{e} de \textit{E} se representará cada estrategia con un valor \textit{a}, siendo \textit{a} el puesto de la estrategia de persuasión en el ranking. Por lo cual, \textit{a} ha de ser siempre menor o igual que el número total de estrategias y mayor que cero, para así mantenerse dentro del rango de \textit{E} (0$<$
\textit{a }$\leq$\textit{ j}), puesto que cada predicción \textit{r} no es más que una mera valoración de un usuario de las \textit{j} estrategias de la lista \textit{E}.
\begin{center}
    \textit{r} = $\begin{bmatrix} \textit{a$_{1}$} & \textit{a$_{2}$} & \textit{a$_{3}$} & \textit{a$_{4}$} & \cdots & \textit{a$_{j}$} \end{bmatrix}$ 
\end{center}
De esta forma se puede afirmar que el primer elemento del ranking tendrá el valor 1 y el último el valor \textit{j}, quedándose las demás estrategias con el resto de los puestos en ese rango. Lo que en un conjunto de 10 estrategias se traduce en que la mejor estrategia obtendría el valor 1 y la peor el valor 10.
\\ \\
Sin embargo, Cuando se trata de realizar varias predicciones se simboliza como una matriz \textit{R} donde cada predicción es representada por \textit{r$_{i}$}, en el que \textit{r} denota la predicción del ranking  e \textit{i} denota la cantidad de predicciones \textit{r} realizadas.
\begin{center}
    \[  \textit{R} = 
        \begin{pmatrix}
            \textit{r$_{1}$}  \\ 
            \textit{r$_{2}$}  \\ 
            \vdots  \\ 
            \textit{r$_{i}$}
        \end{pmatrix} 
        =
        \begin{pmatrix}
            a_{11}  &  a_{12}  &  \cdots   & a_{1j} \\ 
            a_{21}  &  a_{22}  &  \cdots   & a_{2j}\\ 
            \vdots  &  \vdots  &  \ddots & \vdots  \\ 
            a_{i1}  &  a_{i2}  &  \cdots   & a_{ij}
        \end{pmatrix}
    \]
\end{center}
\newpage
\paragraph{Predicción} Una vez entendidas las estructuras básicas que se utilizarán en el método de consenso este apartado presentará su aplicación, ya que en este apartado se crearán las predicciones que serán consensuadas a posteriori para generar los datos de reentrenamiento.
\\ \\
Se debe tener en cuenta que las operaciones anteriores cambian por completo cuando se realizan predicciones para un conjunto \textit{U} de usuarios, y si además estas predicciones se hacen con los diferentes modelos \textit{m} de los participantes las operaciones se complican bastante. 
\\ \\
En primer lugar, se ha de especificar que el conjunto de usuarios \textit{U} estará formado por los \textit{k} identificadores de los usuarios, representados como \textit{u$_{k}$}. Este conjunto de identificadores se crea cuando se generan los datos personales sintéticos y recoge todos los identificadores de todos los usuarios generados aleatoriamente.
\begin{center}
    \textit{U} = $\begin{bmatrix} \textit{u$_{1}$} & \textit{u$_{2}$} & \textit{u$_{3}$} & \textit{u$_{4}$} & \cdots & \textit{u$_{k}$} \end{bmatrix}$ 
\end{center}
Para realizar las distintas predicciones con el modelo \textit{m} de cada participante de la red de aprendizaje federado hay que tener en cuenta que las predicciones \textit{R} pertenecerán al modelo \textit{m}, por lo que se denotará a partir de ahora como \textit{R$_{m}$}. Además, esta matriz de predicciones \textit{R$_{m}$} estará formada por las distintas predicciones para los distintos usuarios \textit{u}, lo que será representado como \textit{r$_{u}$}.
\begin{center}
    \[  \textit{R$_{m}$} = 
        \begin{pmatrix}
            \textit{r$_{1}$}  \\ 
            \textit{r$_{2}$}  \\ 
            \vdots  \\ 
            \textit{r$_{u}$}
        \end{pmatrix} 
        =
        \begin{pmatrix}
            a_{11}  &  a_{12}  &  \cdots   & a_{1j} \\ 
            a_{21}  &  a_{22}  &  \cdots   & a_{2j}\\ 
            \vdots  &  \vdots  &  \ddots & \vdots  \\ 
            a_{u1}  &  a_{u2}  &  \cdots   & a_{uj}
        \end{pmatrix}
    \]
\end{center}
\newpage
\paragraph{Redistribución de las predicciones}
Las predicciones realizadas con cada modelo \textit{m} sobre el conjunto de usuarios \textit{U} puede ser interpretada de diferente forma, en un principio se entiende que cada modelo \textit{m} genera una predicción \textit{R} para cada uno de los \textit{u} usuarios generados, pero, del mismo modo, se puede entender que cada usuario \textit{u} obtiene una predicción \textit{r} por cada modelo \textit{m}, y que con la unión de todas las predicciones \textit{r} de todos los modelos \textit{m} para este mismo usuario \textit{u} se puede construir una matriz \textit{R}. 
\\ \\
Para entender lo anterior con más facilidad se ha de comprender que ya no solo se cuenta con una matriz de predicciones \textit{R}, sino que se cuenta con una matriz \textit{P} con \textit{m} matrices de predicciones \textit{R}, denotada como \textit{P$_{m}$}. Esta matriz representa todas las predicciones \textit{R} realizadas por cada modelo \textit{m} participante en la red de aprendizaje federado para un mismo conjunto de usuarios \textit{U}.
\begin{center}
    \[  \textit{P$_{m}$}=
        \begin{pmatrix}
            \begin{pmatrix}
                a_{11}  &  a_{12}  &  \cdots   & a_{1j} \\ 
                a_{21}  &  a_{22}  &  \cdots   & a_{2j}\\ 
                \vdots  &  \vdots  &  \ddots & \vdots  \\ 
                a_{u1}  &  a_{u2}  &  \cdots   & a_{uj}
            \end{pmatrix}_{\textit{1}}
            & 
            \cdots 
            &
            \begin{pmatrix}
                a_{11}  &  a_{12}  &  \cdots   & a_{1j} \\ 
                a_{21}  &  a_{22}  &  \cdots   & a_{2j}\\ 
                \vdots  &  \vdots  &  \ddots & \vdots  \\ 
                a_{u1}  &  a_{u2}  &  \cdots   & a_{uj}
            \end{pmatrix}_{\textit{m}}
        \end{pmatrix}
    \]
\end{center}
Teniendo en cuenta esta estructura de datos, la separación de los usuarios ha de hacerse tomando cada predicción \textit{r$_{u}$} de cada participante de forma separada, uniendo las que corresponden al mismo usuario \textit{u} en una nueva matriz de predicciones \textit{R}, que al pertenecer al usuario \textit{u} será denotada como \textit{R$_{u}$}. Cada predicción \textit{r} de \textit{R$_{u}$} es extraída de la predicción \textit{r$_{u}$} de cada matriz de predicciones de cada modelo \textit{R$_{m}$}. Además, todas las matrices de predicción \textit{R} de todos los usuarios \textit{u} serán agrupadas bajo el termino \textit{P$_{u}$}. De esta forma puede afirmarse que \textit{P$_{u}$} está formada por las distintas predicciones \textit{R$_{u}$}, las cuales a su vez están formadas por las distintas predicciones \textit{r$_{u}$} extraídas de \textit{R$_{m}$}.
\begin{center}
    \[  \textit{R$_{m}$} = 
        \begin{pmatrix}
            a_{11}  &  a_{12}  &  \cdots   & a_{1j} \\ 
            a_{21}  &  a_{22}  &  \cdots   & a_{2j}\\ 
            \vdots  &  \vdots  &  \ddots & \vdots  \\ 
            a_{u1}  &  a_{u2}  &  \cdots   & a_{uj}
        \end{pmatrix}
        \equiv
        \begin{pmatrix}
            \begin{pmatrix} a_{1}  &  a_{2}  &  \cdots   & a_{j} \end{pmatrix}_{1} \\ 
            \begin{pmatrix} a_{1}  &  a_{2}  &  \cdots   & a_{j} \end{pmatrix}_{2} \\ 
            \vdots \\ 
            \begin{pmatrix} a_{1}  &  a_{2}  &  \cdots   & a_{j} \end{pmatrix}_{u}
        \end{pmatrix}
    \]
    Predicciones del modelo \textit{m} para \textit{u} usuarios.
    \\
    \[  \textit{P$_{m}$}=
        \begin{pmatrix}
            \textit{R$_{1}$} & \textit{R$_{2}$} & \cdots & \textit{R$_{m}$}
        \end{pmatrix}
        \equiv
        \begin{pmatrix}
            \begin{pmatrix}
                \begin{pmatrix} a_{1}  &  a_{2}  &  \cdots   & a_{j} \end{pmatrix}_{1} \\ 
                \begin{pmatrix} a_{1}  &  a_{2}  &  \cdots   & a_{j} \end{pmatrix}_{2} \\ 
                \vdots \\ 
                \begin{pmatrix} a_{1}  &  a_{2}  &  \cdots   & a_{j} \end{pmatrix}_{u}
            \end{pmatrix}_{\textit{1}}
            & 
            \cdots 
            &
            \begin{pmatrix}
                \begin{pmatrix} a_{1}  &  a_{2}  &  \cdots   & a_{j} \end{pmatrix}_{1} \\ 
                \begin{pmatrix} a_{1}  &  a_{2}  &  \cdots   & a_{j} \end{pmatrix}_{2} \\ 
                \vdots \\ 
                \begin{pmatrix} a_{1}  &  a_{2}  &  \cdots   & a_{j} \end{pmatrix}_{u}
            \end{pmatrix}_{\textit{m}}
        \end{pmatrix}
    \] 
    Lista de las predicciones de cada modelo \textit{m} para \textit{u} usuarios.
    \\
    \[  \textit{R$_{u}$} =
            \begin{pmatrix}
                a_{11}  &  a_{12}  &  \cdots   & a_{1j} \\ 
                a_{21}  &  a_{22}  &  \cdots   & a_{2j}\\ 
                \vdots  &  \vdots  &  \ddots & \vdots  \\ 
                a_{m1}  &  a_{m2}  &  \cdots   & a_{mj}
            \end{pmatrix}
            \equiv
            \begin{pmatrix}
                \begin{pmatrix} a_{1}  &  a_{2}  &  \cdots   & a_{j} \end{pmatrix}_{1} \\ 
                \begin{pmatrix} a_{1}  &  a_{2}  &  \cdots   & a_{j} \end{pmatrix}_{2} \\ 
                \vdots \\ 
                \begin{pmatrix} a_{1}  &  a_{2}  &  \cdots   & a_{j} \end{pmatrix}_{m}
            \end{pmatrix}
    \]
    Predicciones para el usuario \textit{u} realizadas por los \textit{m} modelos.
    \\
    \[  
        \textit{P$_{u}$}=
        \begin{pmatrix}
            \textit{R$_{1}$} & \textit{R$_{2}$} & \cdots & \textit{R$_{u}$}
        \end{pmatrix}
        \equiv
        \begin{pmatrix}
            \begin{pmatrix}
                \begin{pmatrix} a_{1}  &  a_{2}  &  \cdots   & a_{j} \end{pmatrix}_{1} \\ 
                \begin{pmatrix} a_{1}  &  a_{2}  &  \cdots   & a_{j} \end{pmatrix}_{2} \\ 
                \vdots \\ 
                \begin{pmatrix} a_{1}  &  a_{2}  &  \cdots   & a_{j} \end{pmatrix}_{m}
            \end{pmatrix}_{\textit{1}}
            & 
            \cdots 
            &
            \begin{pmatrix}
                \begin{pmatrix} a_{1}  &  a_{2}  &  \cdots   & a_{j} \end{pmatrix}_{1} \\ 
                \begin{pmatrix} a_{1}  &  a_{2}  &  \cdots   & a_{j} \end{pmatrix}_{2} \\ 
                \vdots \\ 
                \begin{pmatrix} a_{1}  &  a_{2}  &  \cdots   & a_{j} \end{pmatrix}_{m}
            \end{pmatrix}_{\textit{u}}
        \end{pmatrix}
    \] 
    Lista de las predicciones para cada usuario \textit{u} realizadas por los \textit{m} modelos.
    \\
\end{center}

Para concluir con lo anteriormente explicado, este proceso puede definirse como una transformación de las predicciones que da cada modelo para un grupo de usuarios en las predicciones de cada usuario según cada modelo, \textit{P$_{m}$}$\Longrightarrow $\textit{P$_{u}$}. Además, hay que tener en cuenta que el tamaño de las predicciones de cada usuario depende de la cantidad de modelos que participen en la red de aprendizaje federado, de forma que cuantos más modelos participen más predicciones tendrá cada usuario. 
\\ \\
El proceso de transformación de \textit{P$_{m}$} en \textit{P$_{u}$} explicado previamente y representado mediante las matrices es un proceso que se lleva a cabo en tres partes, primero el de definición de las variables necesarias, segundo el de predecir de los rankings para los \textit{U} usuarios con los \textit{m} modelos y el tercero en el que estas predicciones de los modelos son reordenadas para agruparlas por usuario y no por modelo. El proceso está representado mediante el algoritmo \ref{algth:PreparacionConsenso}.
\\ \\
En este algoritmo la declaración de variables se produce entre las líneas 3 y 9, donde se respeta la nomenclatura utilizada hasta el momento, de forma que, si no se entiende alguna variable se puede acudir a las explicaciones previas para comprender su función. 
\\ \\
Después de esto, se realizan las predicciones de los rankings para todos los usuarios generados sintéticamente, donde \textit{P$_{m}$} está formada por \textit{len(m)} matrices, donde cada matriz \textit{R$_{m}$} representa los valores de ranking \textit{r$_{m}$} que da el modelo \textit{m} para cada usuario \textit{u}.
\\ \\
Una vez habiendo completado las predicciones con todos los modelos y haber obtenido la lista de matrices \textit{P$_{m}$}, es el momento de agrupar los datos por usuarios \textit{u} en vez de por modelos \textit{m}, \textit{P$_{m}$}$\Longrightarrow $\textit{P$_{u}$}. Para este proceso se recorre la lista de usuarios sintéticos \textit{U} y las predicciones \textit{r$_{u}$} que cada modelo da para cada uno de ellos. De esta forma se consigue que \textit{P$_{u}$} este formada por \textit{len(U)} matrices, donde cada matriz \textit{R$_{u}$} representa los valores de ranking \textit{r$_{u}$} que ha dado cada modelo \textit{m} para el usuario \textit{u}. 
\begin{algorithm}[H]
    \caption{Redistribución de las predicciones}\label{algth:PreparacionConsenso}
    \begin{algorithmic}[1]
        \Procedure{Obtener rankings para los usuarios}{}
            \BState \emph{Declaración de las variables necesarias}:
            \State $E \gets [ \dots ]$\Comment{\parbox[t]{0.7\linewidth}{Lista de las estrategias a ordenar}}
            \\
            \State $U \gets [ \dots ]$\Comment{\parbox[t]{0.7\linewidth}{Lista de los identificadores de los usuarios sintéticos}}
            \\
            \State $P_{u} \gets [ len(U) ]$\Comment{\parbox[t]{0.7\linewidth}{Lista de las predicciones de cada usuario}}
            \\
            \State $M \gets [ \dots ]$ \Comment{\parbox[t]{0.7\linewidth}{Lista de los modelos de los participantes}}
            \\
            \State $P_{m} \gets [ len(M) ]$ \Comment{\parbox[t]{0.7\linewidth}{Lista de las predicciones de cada modelo}}
            \\                
            \BState \emph{Predicciones de los modelos}:
            \For {$\textit{int}(participante) \gets 0$ \textbf{to} $\textit{range}(len(M))$}
                \State $R_{m} \gets M[participante].predict(U)$ \Comment{\parbox[t]{0.44\linewidth}{Predecir con el modelo del participante el ranking para los usuarios}}
                \\
                \State $P_{m}[participante] \gets R_{m}$ \Comment{\parbox[t]{0.44\linewidth}{Añadir las predicciones del modelo a la lista de todos los participantes}}
            \EndFor
            \State end \textbf{for};
            \\
            \BState \emph{Redistribución de las predicciones en base a usuarios}:
            \For {$\textit{int}(usuario) \gets 0$ \textbf{to} $\textit{range}(len(U))$}
                \State $R_{u} \gets [\dots]$\Comment{\parbox[t]{0.7\linewidth}{Lista de las predicciones para el usuario}}
                \\
                \For {$\textit{int}(participante) \gets 0$ \textbf{to} $\textit{range}(len(M))$}
                    \State $r_{u} \gets P_{m}[participante][usuario]$ \Comment{\parbox[t]{0.4\linewidth}{Predicción del usuario}}
                    \\
                    \State $R_{u}[participante] \gets r_{u}$ \Comment{\parbox[t]{0.4\linewidth}{Añadir la predicción a la lista de predicciones del usuario}}
                \EndFor
                \State end \textbf{for};
                \\
                \State $P_{u}[usuario] \gets R_{u}$ \Comment{\parbox[t]{0.6\linewidth}{Añadir la lista de predicciones del usuario a la lista de predicciones de todos los usuarios}}
            \EndFor
            \State end \textbf{for};

        \EndProcedure
    \end{algorithmic}
\end{algorithm}

\paragraph{Consensuar predicciones} El proceso de consenso es el proceso clave a la hora de agregar los distintos modelos, ya que es gracias a este proceso que se consiguen los datos para reentrenar los modelos. El proceso parte de la lista \textit{P$_{u}$} formada por \textit{len(U)} predicciones \textit{R$_{u}$}, a las cuales se les aplica una función estadística \textit{func()} sobre los \textit{a$_{j, m}$} elementos de los múltiples rankings para converger en un único valor la diferente lista de valores. Después de aplicar dicha función sobre todas las columnas \textit{j} de la predicción \textit{R$_{u}$} se consigue unificar cada columna \textit{j} en un único valor representado como \textit{b$_{j}$}, con esto se obtiene una única predicción para el usuario \textit{u}, a la cual denotaremos a partir de ahora como \textit{q$_{u}$}. La lista de todas las predicciones consensuadas para el conjunto de usuarios \textit{U} será denotada como \textit{R$_{q}$}.

\begin{center}
    \[  \textit{R$_{u}$}=
        \begin{pmatrix}
            \begin{pmatrix} a_{1}  &  a_{2}  &  \cdots   & a_{j} \end{pmatrix}_{1} \\ 
            \begin{pmatrix} a_{1}  &  a_{2}  &  \cdots   & a_{j} \end{pmatrix}_{2} \\ 
            \vdots \\ 
            \begin{pmatrix} a_{1}  &  a_{2}  &  \cdots   & a_{j} \end{pmatrix}_{m}
        \end{pmatrix}
        \implies
        \begin{pmatrix}
            \begin{pmatrix} a_{1}  \\  a_{2}  \\  \vdots   \\ a_{m} \end{pmatrix}_{1} & 
            \begin{pmatrix} a_{1}  \\  a_{2}  \\  \vdots   \\ a_{m} \end{pmatrix}_{2} & 
            \cdots &
            \begin{pmatrix} a_{1}  \\  a_{2}  \\  \vdots   \\ a_{m} \end{pmatrix}_{j}
        \end{pmatrix}    
    \]
    \\
    \[  \textit{q$_{u}$} = 
        \begin{pmatrix}
            \textit{func}\begin{pmatrix} a_{1}  \\  a_{2}  \\  \vdots   \\ a_{m} \end{pmatrix}_{1} & 
            \textit{func}\begin{pmatrix} a_{1}  \\  a_{2}  \\  \vdots   \\ a_{m} \end{pmatrix}_{2} & 
            \cdots &
            \textit{func}\begin{pmatrix} a_{1}  \\  a_{2}  \\  \vdots   \\ a_{m} \end{pmatrix}_{j}
        \end{pmatrix}   
    \]
    \\
    \[  \textit{q$_{u}$} = 
        \begin{pmatrix}
            \textit{func}\begin{pmatrix} a_{1}  &  a_{2}  &  \cdots   & a_{m} \end{pmatrix}_{1} & 
            \textit{func}\begin{pmatrix} a_{1}  &  a_{2}  &  \cdots   & a_{m} \end{pmatrix}_{2} & 
            \cdots & 
            \textit{func}\begin{pmatrix} a_{1}  &  a_{2}  &  \cdots   & a_{m} \end{pmatrix}_{j}
        \end{pmatrix}
    \]
    \\
    \[  \textit{q$_{u}$} = 
        \begin{pmatrix}
            b_{1}  &  b_{2}  &  \cdots   & b_{j} 
        \end{pmatrix}
    \]
    \\
    \[ 
        \textit{R$_{q}$} = 
        \begin{pmatrix}
            q_{1}  \\  q_{2}  \\  \vdots \\ q_{u}
        \end{pmatrix}
        \equiv
        \begin{pmatrix}
            \begin{pmatrix}
                b_{1}  &  b_{2}  &  \cdots   & b_{j} 
            \end{pmatrix}_{1}\\
            \begin{pmatrix}
                b_{1}  &  b_{2}  &  \cdots   & b_{j} 
            \end{pmatrix}_{2}\\
            \vdots\\
            \begin{pmatrix}
                b_{1}  &  b_{2}  &  \cdots   & b_{j} 
            \end{pmatrix}_{u}
        \end{pmatrix}
    \]
\end{center}    

Las funciones para calcular el valor de consenso \textit{b$_{j}$} que se proponen son las métricas estadísticas de la media(\(\bar{x}\)), mediana(\(\tilde{x}\)) y moda (\(\hat{x}\)). En este proyecto se ha descartado la utilización de la media ya que en caso de que los valores \textit{a$_{j, m}$} fueran muy dispares podría distorsionarse bastante el resultado \textit{b} y perder eficiencia. Por ello solo se utilizarán los algoritmos de mediana(\(\tilde{x}\)) y moda (\(\hat{x}\)).
\begin{center}
    \[  \textit{q$_{u}$} = 
        \begin{pmatrix}
            \widehat{ \begin{pmatrix} a_{1}  &  a_{2}  &  \cdots   & a_{m} \end{pmatrix}_{1}} & 
            \widehat{ \begin{pmatrix} a_{1}  &  a_{2}  &  \cdots   & a_{m} \end{pmatrix}_{2}}& 
            \cdots & 
            \widehat{ \begin{pmatrix} a_{1}  &  a_{2}  &  \cdots   & a_{m} \end{pmatrix}_{j}}
        \end{pmatrix}
    \]
    Representación del consenso por moda
    \\
    \[  \textit{q$_{u}$} = 
        \begin{pmatrix}
            \widetilde{ \begin{pmatrix} a_{1}  &  a_{2}  &  \cdots   & a_{m} \end{pmatrix}_{1}} & 
            \widetilde{ \begin{pmatrix} a_{1}  &  a_{2}  &  \cdots   & a_{m} \end{pmatrix}_{2}}& 
            \cdots & 
            \widetilde{ \begin{pmatrix} a_{1}  &  a_{2}  &  \cdots   & a_{m} \end{pmatrix}_{j}}
        \end{pmatrix}
    \]
    Representación del consenso por mediana
\end{center}    
Con el objetivo de crear una función más precisa de consenso se elaboró un método híbrido el cual combina tanto la métrica de la moda como la de la mediana, algoritmo \ref{algth:Consensuar}. Este método funciona de la siguiente forma: en primer lugar lee la lista de elementos \textit{a$_{j, m}$} y si existe algún valor duplicado más de \textit{len(m)/2} veces automáticamente asigna ese valor como \textit{b$_{j}$}, de lo contrario calculará la mediana de la lista y asignaría ese valor como \textit{b$_{j}$}. Se podrían tomar otras decisiones en función de la desviación típica de la lista e incluso desarrollar un algoritmo más complejo para optimizar este proceso, pero dicha tarea se sale del alcance del proyecto.
\\ \\
Como elemento adicional se ha considerado la posibilidad de otorgar un peso al modelo \textit{m} de cualquier participante en concreto, de forma que este tenga más repercusión en el resultado del consenso. Este peso solo será otorgado al modelo que sea más afín al usuario sintético, lo que puede ser calculado por una función o seleccionado por algún atributo en concreto (país, puesto de trabajo, género, etc.). En este caso este peso se da en función del atributo país, de forma que el modelo del país correspondiente al usuario sintético creado será el que tenga más trascendencia a la hora de elaborar el ranking consensuado. Este peso será representado como \textit{W$_{c}$} y su valor será un número entero de rango 0 $\leq$ \textit{W$_{c}$} $\leq$ \textit{len(m)/2 +1}. Este rango permite no añadir ninguna importancia extra a ningún modelo \textit{m} o permitir que sea este modelo en exclusiva el que genere el ranking para este usuario \textit{u}, ya que debido a la función de consenso, cuando existen \textit{len(m)/2 +1} veces un valor en una lista de elementos \textit{a$_{j, m}$} este valor es asignado automáticamente por ser moda.
\\ \\
Este peso se otorga directamente sobre la predicción \textit{r$_{u}$} del modelo \textit{m} cuando este genera \textit{R$_{m}$}, al extraerse cada predicción \textit{r$_{u}$} de \textit{R$_{m}$} para formar \textit{R$_{u}$} esta predicción se introducirá \textit{W$_{c}$} veces más en \textit{R$_{u}$} para que a la hora de aplicar \textit{func()} esta predicción \textit{r$_{u}$} de \textit{R$_{m}$} tenga mayor importancia respecto a la del resto. De esta forma, como se ha dicho antes, si \textit{W$_{c}$} fuese cero todos los modelos tendrían la misma importancia a la hora de calcular el consenso y si fuera \textit{len(m)/2 +1} este valor sería la moda y por ello sería su valor directamente asignado a \textit{b$_{j}$}.

\begin{algorithm}[H]
    \caption{Consenso de los datos}\label{algth:Consensuar}
    \begin{algorithmic}[1]
        \Procedure{Realización del consenso}{}
            \BState \emph{Declaración de las variables necesarias}:
            \State $P_{u} \gets [ \cdots ]$\Comment{\parbox[t]{0.7\linewidth}{Lista de las predicciones de cada usuario}}
            \\
            \State $R_{q} \gets [\cdots]$\Comment{\parbox[t]{0.7\linewidth}{Lista de rankings consensuados}}
            \\           
            \BState \emph{Consensuar}:
            \For {$\textit{int}(usuario) \gets 0$ \textbf{to} $\textit{len}(P_{u})$}
                \State $R_{u}  \gets P_{u}[usuario]$ \Comment{\parbox[t]{0.44\linewidth}{Lista de predicciones para un mismo usuario}}
                \\
                \State $columnas, filas  \gets shape(R_{u})$ \Comment{\parbox[t]{0.44\linewidth}{Obtener la forma de la matriz de predicciones}}
                \\
                \State $q_{u} \gets [columnas]$\Comment{\parbox[t]{0.44\linewidth}{Inicializar lista de valores para el ranking consensuado}}
                \\
                \For {$\textit{int}(columna) \gets 0$ \textbf{to} $\textit{columnas}$}
                    \State $ r_{i} \gets R_{u}[:,columna]$ \Comment{\parbox[t]{0.44\linewidth}{Lista de todos los valores de la misma columna}}
                    \\
                    \State $ unicos \gets unique(r_{i})$ \Comment{\parbox[t]{0.44\linewidth}{Cantidad de valores diferentes en la columna}}
                    \\
                    \If{$unicos > len(m)/2 $}
                        \State $q_{u}[columna] \gets \textit{mod}(r_{i} )$\Comment{\parbox[t]{0.44\linewidth}{Asignar valor por moda}}
                    \Else
                        \State $q_{u}[columna] \gets \textit{med}(r_{i} )$\Comment{\parbox[t]{0.44\linewidth}{Asignar valor por mediana}}
                    \EndIf
                    \State end \textbf{if};
                \EndFor
                \State end \textbf{for};
                \State $R_{q}[usuario] \gets q_{u}$\Comment{\parbox[t]{0.44\linewidth}{Guardar valor consensuado en }}
            \EndFor
            \State end \textbf{for};
            \State \textbf{return} $R_{q}$;
        \EndProcedure
    \end{algorithmic}
\end{algorithm}
\newpage
\paragraph{Normalización del consenso}
El problema es que en función de la fórmula que se aplique el rango de \textit{q$_{u}$} puede que no permanezca dentro del rango de \textit{E}, lo que nos impide afirmar que el valor \textit{b} de cada estrategia entre dentro del intervalo 0$<$
\textit{a$_{q}$ }$\leq$\textit{ j}. Para solucionar este problema se procesan los valores \textit{a$_{qj}$ } de \textit{q$_{u}$} para que estos permanezcan dentro del rango. Este proceso se realiza intercambiando el valor \textit{a$_{qj}$ } menor por el valor del ranking menor disponible, como se puede ver en el algoritmo \ref{algth:NormalizarConsenso}.

\begin{algorithm}[H]
    \caption{Normalización de los datos del consenso}\label{algth:NormalizarConsenso}
    \begin{algorithmic}[1]
        \Procedure{Normalización del consenso}{}
            \BState \emph{Declaración de las variables necesarias}:
            \State $Indice \gets 1$\Comment{\parbox[t]{0.75\linewidth}{Se inicializa el elemento más importante del ranking como 1}}
            \\
            \State $q_{u} \gets [\cdots]$\Comment{\parbox[t]{0.75\linewidth}{Valores del ranking consensuado}}
            \State $aux \gets q_{u}$\Comment{\parbox[t]{0.75\linewidth}{Variable auxiliar}}
            \\                
            \BState \emph{Normalización del consenso}:
            \For {$\textit{int}(i) \gets 0$ \textbf{to} $\textit{len}(aux)$}
                \State $posicion \gets whereIsMin(aux)$ \Comment{\parbox[t]{0.44\linewidth}{Posición de la lista donde está el menor elemento}}
                \\
                \State $q_{u}[posicion] \gets Indice$ \Comment{\parbox[t]{0.5\linewidth}{Se normaliza el valor del ranking}}
                \\
                \State $\textit{drop}(aux[posicion])$ \Comment{\parbox[t]{0.5\linewidth}{Se elimina el elemento asignado de la lista de los siguientes a normalizar.}}
                \\
                \State $Indice \gets Indice + 1$ \Comment{\parbox[t]{0.5\linewidth}{Se incrementa el índice para asignar la siguiente posición del ranking}}
            \EndFor
            \State end \textbf{for};

            \State \textbf{return} $q_{u}$;

        \EndProcedure
    \end{algorithmic}
\end{algorithm}

\subsubsection{Reentrenado}
Como parte final solo queda reentrenar los modelos de los participantes con los usuarios sintéticos y sus rankings consensuados. Para ello se utilizarán los usuarios y sus atributos generados aleatoriamente antes y los rankings \textit{R$_{u}$} donde cada predicción \textit{r$_{u}$} será el verdadero ranking del usuario.
