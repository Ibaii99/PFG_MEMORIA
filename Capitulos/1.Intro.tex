\chapter{Introducción}
\thispagestyle{fancy}
\fancyhead[LE]{\thechapter.Introducción}

En esta memoria se detalla el estudio sobre la tecnología del Federated Learning, aprendizaje federado, y su implementación en un sistema de recomendación. Esta tecnología es diferente al clásico aprendizaje automático, el cual reúne toda la información en un único dispositivo que se encarga de realizar las operaciones de computación y de creación de los modelos de inteligencia artificial. Gracias al aprendizaje federado se consigue que se puedan crear diferentes modelos de inteligencia artificial distribuidos en los participantes de la red de aprendizaje, de forma que cada participante conserve su información en su dispositivo y que gracias a un mecanismo de agregación global se consiguen agregar los modelos de los diferentes participantes para mejorar el modelo de cada uno de ellos redistribuyendolo. 
\\ \\
En los diferentes apartados se podrá encontrar la siguiente información sobre el proyecto:
\begin{itemize}
    \item \textbf{Antecedentes y justificación}, en esta sección se detallan los acontecimiento y motivaciones que han promovido el desarrollo de este proyecto, donde se comentará el estado del arte de los sistemas de recomendación y se hablará de la justificación del proyecto.
    \item \textbf{Obejtivos y alcance}, apartado en el que se especifican los objetivos que se persiguen en el proyecto tanto en cuanto a la investigación como en cuanto al desarrollo, además de, definir los objetivos que entran dentro del alcance del proyecto y los que no.
    \item \textbf{Consideraciones éticas}, consideraciones éticas tenidas en cuenta para el desarrollo dedl proyecto.
    \item \textbf{Metodología}, hoja de ruta y forma de trabajar que se ha cumplido durante todo el proyecto.
    \item \textbf{Planificación}, listado de tareas y planificación de estas a lo largo del cuatrimestre para la correcta gestión de tiempos en este proyecto.
    \item \textbf{Introducción al Federated Learning}, breve descripción de lo que consiste la tecnología, fundamentos y problemas.
    \item \textbf{Desarrollo}, apartado principal del proyecto, en este se puede encontrar: 
        \begin{itemize}
            \item \textbf{Los requisitos del sistema}, puntos que ha de cumplir y satisfacer el proyecto.
            \item \textbf{Gestión de riesgos del proyecto}, descripción de plan de acción ante problemas previstos.
            \item \textbf{Especificación del diseño}, arquitectura del proyecto, datos involucrados en el sistema de recomendación, protocolo de Federated Learning implementado, securización de las comunicaciones, cómo se agregan los modelos por consenso y la explicación de cómo se valorarán de las predicciones.
            \item \textbf{Experimentos y resultados}, realización de pruebas con el sistema, comparación de los resultados con un sistema de recomendación basado en Machine Learning y las mejoras de rendimiento obtenidas con el Federated Learning.
            \item \textbf{Tecnologías utilizadas}, descripción de las tecnologías, programas y librerías usadas para el desarrollo del proyecto.
            \item \textbf{Consideraciones sobre la implementación}, descripción de las causas que han llevado a diseñar el sistema de la forma en la que ha sido diseñado.
            \item \textbf{Retos abordados en el proyecto}, retos del Federated Learning que se han abordado en el proyecto y de qué forma.
            \item \textbf{Incidencias ocurridas}, problemas con las Raspberries Pi ocurridos durante el desarrollo del proyecto.
        \end{itemize}
    \item \textbf{Presupuesto}, listado de materiales, precio y horas dedicadas al proyecto.
    \item \textbf{Anexo}, información relativa a la protección de datos en la Unión Europea y España.
\end{itemize}

% Antecedentes y justificación
% Objetivos y alcance
% Metodología
% Planificación
% Introducción al Federated Learning
% Desarrollo
%     Requisitos
%     Gestión de riesgos
%     Especificación del diseño
%         Arquitectura
%         Datos involucrados
%         Protocolo de Federated Learning
%         Securización de las comunicaciones
%         Agregación de modelos por consenso
%         Valoración
%     Tecnologías utilizadas
%     Consideraciones del diseño
%     Retos abordados
%     Incidencias
% Presupuesto
