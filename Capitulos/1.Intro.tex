\chapter{Introducción}
\thispagestyle{fancy}
\fancyhead[LE]{\thechapter.Introducción}

En esta memoria se detalla el estudio sobre la tecnología del Federated Learning, aprendizaje federado, y su implementación en un sistema de recomendación. Esta tecnología es diferente al clásico aprendizaje automático, el cual reune toda la información en un único dispositivo que se encarga de realizar las operaciones de computación y de creación de los modelos de inteligencia artificial. Gracias al aprendizaje federado se consigue que se puedan crear diferentes modelos de inteligencia artificial distribuidos en los participantes de la red de aprendizaje, de forma que cada participante conserve su información en su dispositivo y que gracias a un mecanismo de agregación global se consiguen agregar los modelos de los diferentes participantes para mejorar el modelo de cada uno de ellos redistribuyendolo. 
\\ \\
En los diferentes apartados se podrá encontrar la siguiente información sobre el proyecto:
\begin{itemize}
    \item \textbf{Antecedentes y justificación}, en esta sección se detallan los acontecimiento y motivaciones que han promovido el desarrollo de este proyecto.
    \item \textbf{Obejtivos y alcance}, apartado en el que se especifican los objetivos que se persiguen en el proyecto tanto en cuanto a la investigación como en cuanto al desarrollo.
    \item \textbf{Metodología}, hoja de ruta y forma de trabajar que se ha cumplido durante todo el proyecto.
    \item \textbf{Planificación}, listado de tareas y planificación a lo largo del cuatrimestre para la correcta gestión de tiempos en este proyecto.
    \item \textbf{Introducción al Federated Learning}, breve descripción de lo que consiste la tecnología, fundamentos y problemas.
    \item \textbf{Desarrollo}, apartado principal del proyecto, en este se puede encontrar tanto información sobre los requisitos, la gestión de riesgos, la arquitectura del sistema, el sistema de agregación desarrollado, resultados de los experimentos, ...
    \item \textbf{Presupuesto}, listado de materiales, precio y horas dedicadas al proyecto.
    \item \textbf{Anexo}, información relativa a la protección de datos en la Unión Europea y España.
\end{itemize}

% Antecedentes y justificación
% Objetivos y alcance
% Metodología
% Planificación
% Introducción al Federated Learning
% Desarrollo
%     Requisitos
%     Gestión de riesgos
%     Especificación del diseño
%         Arquitectura
%         Datos involucrados
%         Protocolo de Federated Learning
%         Securización de las comunicaciones
%         Agregación de modelos por consenso
%         Valoración
%     Tecnologías utilizadas
%     Consideraciones del diseño
%     Retos abordados
%     Incidencias
% Presupuesto
