\chapter{Consideraciones éticas}
\thispagestyle{fancy}
\fancyhead[LE]{\thechapter.Consideraciones éticas}

En este proyecto se tratan dos consideraciones éticas en mayor medida, la de la defensa del derecho a la privacidad en la era digital y la legalidad para un RS sostenible en el tiempo.

\section{Derecho a la privacidad}

La solución propuesta al problema de agregación de distintas IAs viene de la privacidad como patrón de diseño, motivo y objetivo principal del proyecto. La privacidad digital es un tema controversial a la hora de navegar por internet y en muchas ocasiones, se incumplen derechos fundamentales en el uso de esta red.
\\\\
Tales son las vulneraciones que ocurren, que hay expertos que defienden que no se deben hacer distinciones entre lo digital y lo no digital, ya que vulnerar la privacidad digital es vulnerar el derecho humano de la privacidad. Otros expertos creen que es necesaria una nueva generación de los derechos humanos que incluya los problemas inherentes de vivir en una sociedad digital.
\\\\
En artículos como el “Hacia la cuarta generación de Derechos Humanos: repensando la condición humana en la sociedad tecnológica”\autocite{donasHACIACUARTAGENERACION} del Dr. Javier Bustamante Donas ya se hacía referencia al camino hacia la cuarta generación de los derechos humanos por la tecnologización de la sociedad. En este artículo, el Dr. Javier Bustamante expone varios ejemplos para demostrar la importancia de los derechos digitales. Uno de estos ejemplos es que si se restringe el libre acceso y libre uso de la tecnología se está atentando directamente contra a la libertad de opinión y expresión. Otro claro ejemplo que expresa es el de la censura en China, que es de especial relevancia puesto que afecta a un porcentaje significativo de la sociedad. El caso es que China ha implantado barreras informáticas que impiden la consulta y la visualización de cualquier tipo de páginas web no autorizadas por el gobierno. Además, todo ciudadano chino debe completar un formulario exhaustivo antes de acceder a internet, haciéndolo fácilmente identificable en la red.
\\\\
Con el objetivo de respetar los derechos de los usuarios se tomarán como punto de partida dos declaraciones de derechos digitales con especial relevancia en el contexto de este proyecto, la carta sobre la privacidad digital ded España \autocite{CartaDerechosDigitales} y la declaración de Deusto sobre los Derechos Humanos en Entornos Digitales \autocite{DeclaracionDeustoDerechos}.
\\\\
En cuanto a la declaración de Deusto de los Derechos Humanos en Entornos Digitales, cabe destacar que entre estos derechos los más destacables a cumplir se encuentran los siguientes:
\begin{itemize}
    \item \textit{``Derecho a la transparencia y responsabilidad en el uso de algoritmos''}, ya que toda persona podrá conocer en todo momento como funciona el protocolo de FL y tendrá el control total sobre sus datos.
    \item  \textit{``Derecho a la privacidad en entornos tecnológicos''}, puesto que toda persona al tener el control sobre sus datos puede elegir cuales se usan en la red de FL y de no querer participar puede renegar de ello.
\end{itemize}

España se encuentra redactando una carta sobre la privacidad digital que tendrá un gran impacto en la sociedad y en internet. Supone un hito en Europa y en el mundo que un estado se comprometa a la protección de los derechos digitales de sus ciudadanos (Anexo.\ref{appendix:ProteccionDatos}, \ref{appendix:ProteccionDatos_Derechos} Derechos fundamentales) y como en este trabajo se quieren respetar estos derechos, se ha elaborado el proyecto partiendo del respeto de estos derechos y de la privacidad como patrón de diseño. Debido a que esta carta aún no esta aprobada pero cuando se apruebe será legalmente vinculante se comentará con más detalle en la sección siguiente de legalidad.

\section{Legalidad}
Según el principio de legalidad del código ético y deontológico de la Ingeniería Informática \autocite{CodigoEticoDeontologicoa}, artículo 7, punto 5, todo sistema ha de cumplir tanto con las legislaciones Españolas como Europeas. 
\\\\
Para cumplir con el Reglamento General de Protección de Datos, RGPD (Anexo.\ref{appendix:ProteccionDatos}, \ref{appendix:ProteccionDatos_Legislacion} Legislación ), el propio dispositivo del usuario tiene en su poder el modificar sus datos en caso de que sean inexactos o eliminarlos si desea borrar su huella digital. Por lo cual siempre estará en su mano y bajo su responsabilidad el cómo se utilizan sus datos.
\\\\
Además, con este sistema se pretende cumplir tanto con la legislación vigente como con la pionera carta de derechos digitales de España (Anexo.\ref{appendix:ProteccionDatos}, \ref{appendix:ProteccionDatos_Derechos} Derechos fundamentales) que entrará en vigor en los próximos años. 
\\\\
En esta carta existen dos puntos de gran impacto para este proyecto: los derechos ante la Inteligencia Artificial (Derecho XXIII) y el derecho a no ser localizado y perfilado (Derecho V). Ambos explicados en el apartado Derechos fundamentales \ref{appendix:ProteccionDatos_Derechos} del Anexo.\ref{appendix:ProteccionDatos}.
\begin{itemize}
    \item En cuanto al perfilado de usuarios, es la propia tecnología (FL) la que obliga a su consentimiento, ya que la tecnología parte de la premisa de que los usuarios que participan en la red participan de mutuo acuerdo. Es decir, ningún tipo de red que implemente el FL podrá incumplir esta ley puesto que para entrar en la red hace falta expresarlo explícitamente.
    \item Sobre los derechos ante la IA el usuario tendrá derecho en todo momento a impugnar la recomendación e incluso a corregirla si así lo desea.
\end{itemize}

