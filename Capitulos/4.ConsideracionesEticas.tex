\section{Consideraciones éticas}

En este proyecto surgen dos grandes consideraciones éticas, la de la legalidad y la de la defensa del derecho a la privacidad en la era digital.

\subsection{Legalidad}

Según el principio de legalidad del código ético y deontológico de la Ingeniería Informática \autocite{CodigoEticoDeontologicoa}, artículo 7, punto 5, el sistema ha de cumplir tanto con las legislaciones Españolas como Europeas. 

La solución propuesta al problema de agregación de distintas inteligencias artificiales viene de la privacidad como patrón de diseño, motivo y objetivo principal del proyecto. Con este sistema se pretende cumplir tanto con la legislación vigente como con la pionera carta de derechos digitales de España (Anexo\ref{appendix:ProteccionDatos}, Derechos fundamentales\ref{appendix:ProteccionDatos_Derechos}) que entrará en vigor en los próximos años. 

En esta carta existen dos puntos de gran impacto para este proyecto: los derechos ante la Inteligencia Artificial (Derecho XXIII) y el derecho a no ser localizado y perfilado (Derecho V). Ambos explicados en el apartado de derechos fundamentales del Anexo.

Respondiendo a los derechos en cuanto al perfilado de usuarios, es la propia tecnología (federated learning) la que obliga a su consentimiento, ya que la tecnología parte de la premisa de que los usuarios que participan en la red participan de mutuo acuerdo. Es decir, ningún tipo de red que implemente el federated learning podrá incumplir esta ley puesto que para entrar en la red hace falta expresarlo explícitamente.

Sobre los derechos ante la inteligencia artificial el usuario tendrá derecho en todo momento a impugnar la recomendación e incluso a corregirla si así lo desea.

Para cumplir con la RGPD(Explicado en el apartado de legislación del Anexo) el propio dispositivo del usuario tiene en su poder el modificar sus datos en caso de que sean inexactos o eliminarlos si desea borrar su huella digital. Por lo cual siempre estará en su mano y bajo su responsabilidad el cómo se utilizan sus datos.


\subsection{Derechos a la privacidad}
Cabe destacar como un pequeño apunte que el tema de la privacidad digital es muy controversial, hay expertos que defienden que la privacidad es la privacidad y no se deben hacer distinciones entre lo digital y lo no digital, otros sin embargo, hacen referencia a que es necesaria una nueva generación de los derechos humanos.

En artículos como el “Hacia la cuarta generación de Derechos Humanos: repensando la condición humana en la sociedad tecnológica”\autocite{donasHACIACUARTAGENERACION} del Dr. Javier Bustamante Donas ya se hacía referencia al camino hacia la cuarta generación de los derechos humanos por la tecnologización de la sociedad. En este artículo, el Dr. Javier Bustamante expone varios ejemplos para demostrar la importancia de los derechos digitales. Uno de estos ejemplos es que si se restringe el libre acceso y libre uso de la tecnología se está atentando directamente contra a la libertad de opinión y expresión. Otro claro ejemplo que expresa es el de la censura en China, que es de especial relevancia puesto que afecta a un porcentaje significativo de la sociedad. El caso es que china ha implantado barreras informáticas que impiden la consulta y la visualización de cualquier tipo de páginas web no autorizadas por el gobierno. Además, todo ciudadano chino debe completar un formulario exhaustivo antes de acceder a internet, haciéndolo fácilmente identificable en la red.

Sea como fuere, en estos momentos España se encuentra redactando una carta sobre la privacidad digital que tendrá un gran impacto en la sociedad y en internet. Supone un hito en Europa y en el mundo que un estado se comprometa a la protección de los derechos digitales de sus ciudadanos, para más información me remito al apartado Derechos fundamentales del Anexo.
