\section{Retos abordados en el proyecto}
El aprendizaje federado es una tecnología nueva que con muchas vías de investigación, el objetivo de esta sección es explicar las vías abiertas que se han estudiado y abordado. 
\subsection{Convergencia del algoritmo de aprendizaje}
Una de las consideraciones principales del Federated Learning es la convergencia de los distintos modelos. Se trata de optimizar los modelos, minimizando los datos que se suben a estos. No obstante, la convergencia no está siempre garantizada, pero ya se cuenta con proyectos dedicados a lograr la plena convergencia del modelo global.
\\ \\
Este proyecto trata de abordar este reto desde otro punto de vista, en vez de intentar combinar los algoritmos tratar de que estos compartan el conocimiento a través de usuarios sintéticos como se ha explicado anteriormente. Esta solución, además, permite que este enfoque se pueda aplicar a cualquier caso de uso debido a que no tiene dependencias respecto al framework de inteligencia artificial que se esté usando. Los algoritmos clásicos de FedAvg y FedSGD están implementados en Tensorflow, pero si se quisieran usar en otro sistema habría que desarrollarlos o buscar alguna solución ajena al framework, mientras que el algoritmo de consenso es un sistema que permite converger los resultados de una matriz de predicciones en una simple lista. Que a su vez esta lista será usada para reentrenar a los participantes y mejorar su precisión. Concepto que puede ser aplicado a multitud de sistemas de recomendación.

\subsection{Privacidad}
El reto de la privacidad es cada vez más relevante en un mundo donde todo es cada vez más digitalizado. El Federated Learning cuenta con la privacidad como una de sus principales ventajas, pudiendo utilizar el modelo de aprendizaje de manera local, intercambiando únicamente los parámetros necesarios con el modelo global alojado en el servidor. 
\\ \\
No obstante, un atacante puede obtener acceso a cierta información confidencial si logra acceder al modelo global. Para cubrir esta carencia, se proponen diferentes soluciones en las que se sacrifica algo de rendimiento con el fin de lograr la plena seguridad, y se está logrando un equilibrio mayor con el paso del tiempo.
\\ \\
Este reto ha sido solucionado mediante el enfoque de reentrenado en vez de el clásico enfoque de converger los modelos. Esto quiere decir que, en este caso, los modelos no se \textit{``fusiona''}, sino que cada uno se reentrena con datos sintéticos para los cuales se ha estimado un resultado a través del método de consenso. De esta forma, se evita que un participante albergue en su modelo información de otro participante y podrá evitar que su información este en los dispositivos de otros.

\subsection{Seguridad en la comunicación}
Un reto al que se ha tenido que enfrentar el Federated Learning ha sido la exposición a ciberataques, principalmente mediante interferencias o DoS. El ataque por interferencias consiste en enviar señales de alta frecuencia para generar interferencias en las comunicaciones entre los dispositivos móviles y el servidor, algo que provoca errores en los datos transferidos entre los clientes y el servidor, y que reduce de forma drástica el rendimiento del modelo. 
\\ \\
Sin embargo, este reto se puede abordar de manera exitosa enviando copias del modelo a través de diferentes frecuencias para reducir el riesgo de que el modelo se corrompa de algún modo.
\\ \\
En este proyecto la seguridad de la comunicación ha sido un punto fundamental, ya que todas las comunicaciones son cifradas con cifrado híbrido, y en caso de que un participante no recibiese su modelo por desconexión podría pedirlo de nuevo al nodo principal. Sin embargo, no se ha tenido en cuenta la conexión inalámbrica ni los ataques de denegación de servicio, esto sería un nuevo campo a estudiar y no quedaría resuelto del todo.  

\subsection{Combinación de algoritmos de reducción de comunicaciones}
Actualmente, hay tres técnicas de reducción de comunicaciones en Federated Learning, cuya combinación puede resultar realmente efectiva a la hora de reducir el tamaño de los datos intercambiados con el modelo global. Por ahora, se están desarrollando técnicas que permitan una buena combinación de estos algoritmos con la finalidad de maximizar la eficiencia del sistema global de Federated Learning.
\\ \\
Con el objetivo de maximizar la eficiencia de las comunicaciones se han implementado mecanismo de compresión de los bytes del modelo y de la información a enviar por la red, reduciendo con zlib considerablemente el tamaño de los mensajes.

