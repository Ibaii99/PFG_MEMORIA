\chapter*{Resumen}
%Esta función determina el comienzo de un nuevo capítulo. El parámetro indica el título de lo que será el capítulo. Hay dos funciones que inician un nuevo capítulo "\Chapter*" y "\Chapter", con asterisco y sin él. El asterisco se utiliza para que no sea vea: "Capítulo 1: Resumen" y únicamente se vea "Resumen" (Gustos personales).

\thispagestyle{fancy}
%Como ya se ha visto anteriormente, esta función determina que en las páginas se utilice el estilo fancy (previamente creado). Este estilo no tiene ningún encabezado (como se ha determinado) y sólo incluye los números de página en los pies de página correspondiente (izquierda o derecha).

%%%%%%%%%%%%%%%%%%%%%%%%%%%%%%%%%%%%%%%%%%%%%%%%%%%%%%
%%% A continuación, al final, COMENZAMOS A ESCRIBIR%%% 
%%%%%%%%%%%%%%%%%%%%%%%%%%%%%%%%%%%%%%%%%%%%%%%%%%%%%%


%En LATEX se escribe igual que en cualquier otro procesador de texto, y en este caso sin los símbolos de "%" que estoy utilizando actualmente y que generan comentarios explicativos.%

Lorem ipsum dolor sit amet, consectetur adipiscing elit. Aenean semper non orci at fringilla. Etiam fermentum in diam vestibulum pellentesque. Nam volutpat, velit ut euismod mollis, ipsum erat facilisis justo, et tempor est lectus nec massa. Pellentesque habitant morbi tristique senectus et netus et malesuada fames ac turpis egestas. Sed accumsan viverra neque eu blandit. Suspendisse in fermentum felis, id iaculis mi. Quisque maximus quam lacus, non lobortis libero tincidunt at. Pellentesque habitant morbi tristique senectus et netus et malesuada fames ac turpis egestas. Maecenas lectus nibh, sagittis ut felis non, fermentum ullamcorper leo. In hac habitasse platea dictumst. Orci varius natoque penatibus et magnis dis parturient montes, nascetur ridiculus mus. Fusce est dui, dapibus ultricies ipsum quis, ultrices maximus dolor. Proin finibus, lorem vulputate maximus convallis, quam tellus posuere magna, sed maximus nunc eros quis ipsum. Cras dignissim cursus lectus ac auctor. Donec suscipit vestibulum neque, sed lobortis est rhoncus non.\\

Praesent ut neque eros. Praesent vitae augue at diam tincidunt sagittis. In cursus lorem nec neque condimentum dictum. Morbi vel tristique orci. Cras luctus tempus elit tincidunt hendrerit. Proin bibendum arcu et sapien finibus vulputate sed eget leo. Duis at bibendum massa, sit amet ornare lorem. Donec dictum finibus fringilla. Duis accumsan lectus dolor, eu maximus ligula semper vel. Proin a sem tincidunt, mollis magna eget, consequat tortor. In sodales justo et varius scelerisque. Aliquam sit amet lacus mollis, tincidunt erat vitae, lacinia sem. Maecenas vel erat sagittis, semper ex in, commodo libero.\\

Integer malesuada quis elit eu eleifend. Suspendisse non vestibulum est, a fermentum urna. Donec ligula tortor, ultrices varius nisi eget, dictum malesuada augue. Fusce lacus orci, eleifend quis luctus dictum, ultricies id turpis. Vestibulum et auctor orci. Pellentesque habitant morbi tristique senectus et netus et malesuada fames ac turpis egestas. Integer facilisis neque quis dui dapibus dictum.\\

%IMPORTANTE: LATEX no tiene en cuenta si en el código que habéis escrito ponéis un único espacio entre las palabras o como yo ahora,    256              espacios. LATEX lo procesará como si existiese un único espacio. 




%Lo mismo ocurre con los "intros" para cambiar de párrafo, da igual cuantos escribáis a mano, LATEX interpretará como si sólo hubiera uno. También habréis podido deducir que el "intro" no se escribe con la tecla clásica sino mediante la escritura de dos "\\" al final de cada párrafo.

\textbf{Descriptores}\\
Lorem, ipsum, dolor, sit, amet
